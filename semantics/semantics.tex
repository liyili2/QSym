\documentclass[runningheads]{llncs}

\usepackage{proof}
\usepackage{stmaryrd}
\usepackage{amsmath}
\usepackage{amssymb}
\usepackage{listings}
\usepackage{varwidth}
\usepackage{tikz}
\usepackage{tikz-cd}
\usepackage{nameref}
\usepackage{caption}
\usepackage{subcaption}
\usepackage{empheq}
\usepackage[most]{tcolorbox}
% \usepackage{syntax}
% \usepackage{simplebnf}

\usetikzlibrary{backgrounds,fit,decorations.pathreplacing,calc}
\lstset{basicstyle=\ttfamily, mathescape=true, literate={~} {$\sim$}{1}}

\definecolor{lightgreen}{HTML}{90EE90}

% Based on https://tex.stackexchange.com/a/288326/186240
\newtcbox{\condition}[1][]{%
    nobeforeafter, math upper, tcbox raise base,
    enhanced, colframe=blue!30!black,
    colback=lightgreen, boxrule=1pt,
    #1}

\newcommand {\ite} {\textsf{if}}
\newcommand {\Ra} {\Rightarrow}
\newcommand {\deref} {\textsf{deref}}
\newcommand {\apply} {\textsf{apply}}
\newcommand {\ra} {\rightarrow}
\newcommand {\labinfer} [3] [] {\infer[{\textsc{#1}}]{#2}{#3}}
\newcommand {\Type} {\textsf{Type}}
\newcommand {\Nat} {\textsf{Nat}}
\newcommand {\Bool} {\textsf{Bool}}
\newcommand {\Vect} {\textsf{Vec}}
\newcommand {\ok} {\text{ type}}
\newcommand {\Env} {\textsf{Env}}
\newcommand {\State} {\textsf{State}}
\newcommand {\Value} {\textsf{Value}}
\newcommand {\Locus} {\textsf{Locus}}
\newcommand {\Loc} {\textsf{Location}}
\newcommand {\sem} [1] {\llbracket #1 \rrbracket}
\newcommand {\semE} [1] {\sem{#1}^e}
\newcommand {\Psem} [1] {\sem{#1}^p}
\newcommand {\Tsem} [1] {\sem{#1}^t}
\newcommand {\Expr} {\textsf{Expr}}
\newcommand {\Cmd} {\textsf{Stmt}}
\newcommand {\CmdTy} {\texttt{Stmt}}
\newcommand {\ExprTy} {\texttt{ExprTy}}
\newcommand {\Apply} {\textsf{Apply}}
\newcommand {\fst} {\ensuremath{\textsf{proj}_1}}
\newcommand {\snd} {\ensuremath{\textsf{proj}_2}}
\newcommand {\Var} {\textsf{Var}}
\newcommand {\Index} {\textsf{Index}}
\newcommand {\Cast} {\textsf{Cast}}
\newcommand {\Rename} {\textsf{Rename}}
\newcommand {\Permute} {\textsf{Permute}}
\newcommand {\loceq} {\approx}
\newcommand {\sep} {\ast}
\newcommand {\pointsto} {\mapsto}

\newcommand {\Mem} {\textsf{Mem}}

\newcommand {\ket}[1]{\ensuremath{\left|#1\right\rangle}}

% IR
\newcommand {\ir} [1] {\textsf{\textbf{#1}}}
\newcommand {\IRep} {\textsf{IR}}
\newcommand {\irUpgrade} {\ir{upgrade}}
\newcommand {\irSum} {\ir{sum}}
\newcommand {\irOverwrite} {\ir{overwrite}}
\newcommand {\irSlice} {\ir{slice}}
\newcommand {\irGetBit} {\ir{getBit}}
\newcommand {\irNat} {\ir{nat}}
\newcommand {\irAmp} {\ir{amp}}
\newcommand {\irPhase} {\ir{phase}}
\newcommand {\irBV} {\ir{bitvec}}
\newcommand {\irOmega} {\ir{omega}}

\newcommand {\qafnyUpgrade} {\textnormal{upgrade}}

\newcommand {\updateMem} {\textsf{updateMem}}

\newcommand {\truthVals} {\Omega}

\newcommand {\Qafny} {\textsf{Qafny}}

\begin{document}

\title{QSym Semantics}
\author{}
\institute{}
\maketitle

\section{Overview}

In the following, we define a semantics for QSym by providing a function which takes syntactic QSym terms and produces a function from an ``environment'' to a intermediate representation term. This intermediate representation term can then be transformed into either a traditional linear transformation or into a logical predicate. This logical predicate can be fed into an SMT solver. (Fig.~\ref{fig:sem-fns})

\begin{figure}
  % https://q.uiver.app/#q=WzAsNCxbMCwxLCJcXHRleHRub3JtYWx7UVN5bX0iXSxbMiwxLCJcXHRleHRub3JtYWx7SVJ9Il0sWzQsMCwiXFx0ZXh0bm9ybWFse0xpbmVhciB0cmFuc2Zvcm1hdGlvbn0iXSxbNCwyLCJcXHRleHRub3JtYWx7TG9naWNhbCBwcmVkaWNhdGV9Il0sWzAsMSwiXFxsbGJyYWNrZXQtXFxycmJyYWNrZXQiXSxbMSwyLCJcXGxsYnJhY2tldC1cXHJyYnJhY2tldF50Il0sWzEsMywiXFxsbGJyYWNrZXQtXFxycmJyYWNrZXRecCIsMl1d
  \[\begin{tikzcd}
          &&&& {\textnormal{Linear transformation}} \\
          {\textnormal{Qafny}} && {\textnormal{IR}} \\
          &&&& {\textnormal{Logical predicate}}
          \arrow["{\llbracket-\rrbracket}", from=2-1, to=2-3]
          \arrow["{\llbracket-\rrbracket^t}", from=2-3, to=1-5]
          \arrow["{\llbracket-\rrbracket^p}"', from=2-3, to=3-5]
  \end{tikzcd}\]
  \caption{Outline of semantic functions}
  \label{fig:sem-fns}
\end{figure}


\section{Intermediate representation syntax}

QSym internally uses an intermediate representation. A ``abstract sum'' in the intermediate representation has the following syntactic form:

\[
  \irSum_{\overline{k=a}}^{\overline{k=b}}
    \left(\begin{aligned}
      &e_1\\
      &e_2\\
      &e_3
    \end{aligned}\right)
\]

This is the main syntactic form of the IR. Each of the $\overline{k=a}$ at the bottom gives a lower bound for an index called $k$. The corresponding part at the top gives the upper bounds.
The three expressions are the amplitude, phase and bitvector of the vector represented by the sum body. The entire sum describes the vectors of the new memory state. $a$ and $b$ both must be natural number literals.

The syntax of IR expressions consist of standard arithmetic operations together with the following additional operations:

\begin{itemize}
  \item $\irOmega(j, k)$: This gives roots of unity.
  \item $\irOverwrite(j, a, b)$: This overwrites the bitvector $a$ with the bitvector $b$, starting at index $j$.
  \item $\irSlice(a, j, k)$: This gets the sub-bitvector from bitvector $a$ starting at index $j$ and ending at index $k$.
  \item $\irNat(a)$: Interpret the bitvector $a$ as a natural number.
  \item $\irAmp(v)$: Get the amplitude for vector $v$.
  \item $\irPhase(v)$: Get the phase for vector $v$.
  \item $\irBV(v)$: Get the bitvector for vector $v$.
  \item $\irUpgrade(A, a, b)$: Upgrade the operation $A$ so that it has an additional index with bounds $[a,b]$.
\end{itemize}

This intermediate representation has two sets of semantics, given by the linear transformation interpretation and the logical predicate interpretation. Because of this, we also allow the embedding of elements of another arbitrary set into the syntax of the IR. This is similar to how elements of $\mathbb{Z}$ can often be embedded into a language as an integer literal.

\section{Language of logical predicates}

This is first-order logic, together with the special operations mentioned in the IR syntax section.

\section{Translation into IR}

Recall that we will later give two different semantics for IR terms. Each of these will have a different sort of environment. With that in mind, we will say we are given some set $A$ which will serve as a representation of our vectors. Later, we will choose an appropriate set for each of the translations. We will also have a locus environment, $\Env_A$.

\textit{(\textbf{TODO}: More on locus environment.)}

So, we define a semantic function $\sem{-} : \Qafny \ra (\Env_A \times A \ra \IRep)$. We start by defining its output when it's applied to the $H$ (Hadamard) operation:

\[
  \sem{\ell := A}(\sigma, v) = \semE{A}(\sigma(\ell), v)
  %
  %
  % \irSum_{k=1}^{k=2}
  %   \left(\begin{aligned}
  %     &\frac{1}{\sqrt{2}} * \irAmp(v)\\
  %     &\irOmega(\irGetBit(\irBV(v), \sigma(\ell)), 2^N)\\
  %     &\irOverwrite(\sigma(\ell), \irBV(v), k)
  %     \end{aligned}\right)
\]

\[
  \semE{H}(u, v) = 
  \irSum_{k=1}^{k=2}
    \left(\begin{aligned}
      &\frac{1}{\sqrt{2}} * \irAmp(v)\\
      &\irOmega(\irGetBit(\irBV(v), u), 2^N)\\
      &\irOverwrite(u, \irBV(v), k)
      \end{aligned}\right)
\]

\[
  \semE{\qafnyUpgrade(A, a, b)}(u, v) = \irUpgrade(\semE{A}, a, b)(u, v)
\]

where $\irGetBit$ is defined as $\irGetBit(a, j) = \irSlice(a, j, j)$.

\section{Linear transformation semantics of IR}

Here, we define a translation $\Tsem{-} : \IRep \ra (\Env \times \mathbb{C}^N \ra \mathbb{C}^N)$. We assume that the number of qubits is fixed as N.

\[
  \Tsem{\irSum_{\overline{k=a}}^{\overline{k=b}}
    \left(\begin{aligned}
      &e_1\\
      &e_2\\
      &e_3
    \end{aligned}\right)}(\sigma, v)
    =
    \sum_{j=1}^{2^N}\sum_{\overline{k=a}}^{\overline{k=b}}
      (\underbrace{\Tsem{e_1}(v_j, \overline{k})}_{\textnormal{Amplitude}} \cdot \underbrace{\Tsem{e_2}(v_j, \overline{k})}_{\textnormal{Phase}} \cdot \underbrace{\ket{\Tsem{e_3}(v_j, \overline{k})}}_{\textnormal{Bitvector}})
\]

\[
  \Tsem{\irUpgrade(A, a, b)}(\sigma, v) =
    \sum_{j=a}^{j=b} \Tsem{A}(\sigma, v_j)
\]

\noindent
The standard mathematical operations translate directly:

\[
  \Tsem{a + b}\sigma = \Tsem{a}\sigma + \Tsem{b}\sigma
\]

\section{Logical predicate semantics of IR}

The environment here is pairs of memory states. The first item is the old memory state and the second is the new memory state. Specifically, we have this where $\truthVals$ represents the set of truth values for the logic: $\Psem{-} : \IRep \ra (\Env \times (\Mem \times \Mem) \ra \truthVals)$.

If $m$ is a memory state, then $m[\overline{k}]$ represents the vector at the indices $\overline{k}$. The amplitude of the (``logical'') vector $u$ is given by $u_1$, the phase is given by $u_2$ and the bitvector is given by $u_3$.

\begin{multline*}
  \Psem{\irSum_{\overline{k=a}}^{\overline{k=b}}
    \left(\begin{aligned}
      &e_1\\
      &e_2\\
      &e_3
    \end{aligned}\right)}(\sigma, (m, m')) =\\
    \forall j. \forall \overline{k}. (j >= 1) \land (j <= 2^n) \land (\overline{k >= a} \land \overline{k <= b})\\
    \implies \updateMem(m, m', j, \overline{k}, e_1, e_2, e_3)
    % m'(j)_1 = \Psem{e_1}(m) \land
    % m'(j)_2 = \Psem{e_2}(m) \land
    % m'(j)_3 = \Psem{e_3}(m)
\end{multline*}

\[
\begin{aligned}
  \Psem{\irUpgrade(A, &a, b)}(\sigma, (m, m')) =\\
    &\forall j. (j >= a) \land (j <= b)\\
      &\implies \Psem{A}(m[j], m'[j])
\end{aligned}
\]

We define $\updateMem$ as follows.

\[
\begin{aligned}
  \updateMem(m, m', j, &\overline{k}, e_1, e_2, e_3) =\\
    &\left(m'[\overline{k}] = \left(\begin{aligned}
                          &\Psem{e_1}(m[j], \overline{k})\\
                          &\Psem{e_2}(m[j], \overline{k})\\
                          &\Psem{e_3}(m[j], \overline{k})
    \end{aligned}\right)\right)
\end{aligned}
\]

\noindent
IR expressions are translated as follows

\begin{align*}
  &\Psem{\irAmp(e)}(v, \overline{k}) = (\Psem{e}(v, \overline{k}))_1\\
  &\Psem{\irPhase(e)}(v, \overline{k}) = (\Psem{e}(v, \overline{k}))_2\\
  &\Psem{\irBV(e)}(v, \overline{k}) = (\Psem{e}(v, \overline{k}))_3
\end{align*}

When $x$ is the variable representing the old vector, we have $\Psem{x}(v, \overline{k}) = v$. When $y$ is a variable representing the index variable $k_a$ from $\overline{k}$, we have $\Psem{x}(m, \overline{k}) = k_a$.

\end{document}

